\section{Bakgrund}

Utvecklingen av webben har gett oss i tur och ordning HTML, CSS och JavaScript. JavaScript introducerades 1995 och var främst avsett att lösa små uppgifter som validering av indata och enklare animationer \parencite{Moller2018}. På senare år har webben utvecklats till att utöver vara en plattform för webbplatser även vara en plattform för (webb-)appar. Samtidigt har tillverkare av webbläsare de senaste åren fokuserat på att optimera prestanda vid körning av JavaScript på olika sätt, exempelvis genom att introducera ''Just-In-Time'' (JIT) kompilatorer \parencite{HerreraChenLavoieHendren2018}. Enligt \textcite{ReiserBlaser2017} finns det dock alltid ett önskemål om högre prestanda och \textcite{Zakai2018} beskriver JavaScript som ett hinder för krävande (eng. high-performance) appar. WebAssembly avses enligt \textcite{HaasRossbergSchuffTitzerHolmanGohmanWagnerZakaiBastien2017} vara ett alternativ till JavaScript med högre prestanda. Högre prestanda ger utrymme för mer krävande webb-appar.

WebAssembly\footnote{https://webassembly.org/} (Wasm) är enligt \textcite{HaasRossbergSchuffTitzerHolmanGohmanWagnerZakaiBastien2017} ett portabelt assembly-språk för webbläsare och i framtiden andra tillämpningar. Tekniskt sett är WebAssembly mellankod (eng. bytecode) som skapats genom att \emph{kompilera} (valfri form av) kod till WebAssembly \parencite{Watt2018} som sedan körs i en WebAssembly-motor, detta kan jämföras med JavaScript-kod som traditionellt sett \emph{tolkas} allt eftersom den körs i en JavaScript-motor. WebAssembly kan ses som en ersättare till tidigare tekniker som asm.js från Mozilla och Native Client (NaCl) från Google. NaCL kör nativ (eng. native) kod i en avgränsad del i Chrome, medan asm.js \parencite{Zakai2018} är en delmängd av JavaScript optimerad för prestanda \parencite{VanEsNicolayStievenartDHondtDeRoover2016} och kan köras i alla webbläsare.

Tidigare arbete kring vad som idag är WebAssembly har också resulterat i Emscripten \parencite{Zakai2011}, en kompilator baserad på LLVM \parencite{LattnerAdve2014} som ursprungligen togs fram för att kompilera JavaScript till asm.js \parencite{Zakai2011} men som har vidareutvecklats \parencite{HaasRossbergSchuffTitzerHolmanGohmanWagnerZakaiBastien2017} och nu kan kompilera både JavaScript och C/C++ till både asm.js och WebAssembly. Emscripten är den vanligast förekommande kompilatorn för att kompilera WebAssembly.

WebAssembly är resultatet av gemensam forskning och utveckling mellan  Apple, Google, Microsoft och Mozilla \parencite{HaasRossbergSchuffTitzerHolmanGohmanWagnerZakaiBastien2017} och WebAssembly har därför som första teknik sedan JavaScript \emph{fullt} formellt stöd i Chrome, Edge, Firefox och Safari \parencite{Moller2018}. De webbläsare som inte stödjer WebAssembly kan enligt \textcite{HaasRossbergSchuffTitzerHolmanGohmanWagnerZakaiBastien2017} använda asm.js  som ''polyfill''. WebAssembly används redan idag där prestanda är av vikt, exempelvis för att generera kryptovaluta \parencite{RuthZimmermannWolsingHohlfeld2018}.

Initialt stöder WebAssembly kod skriven i C/C++ \parencite{HaasRossbergSchuffTitzerHolmanGohmanWagnerZakaiBastien2017}. Detta fokus är till största delen ett resultat av begränsningen att WebAssembly ännu inte har stöd för automatisk minneshantering (eng. garbage collection). Enligt \textcite{HaasRossbergSchuffTitzerHolmanGohmanWagnerZakaiBastien2017} så är det ett högt prioriterat mål att från WebAssembly tillåta åtkomst till webbläsarens inbyggda minneshanterare för att på så sätt stödja språk som använder sig av automatisk minnehantering.

WebAssembly laddas in som en modul via ett JavaScript API eller en annan WebAssembly-modul \parencite{HaasRossbergSchuffTitzerHolmanGohmanWagnerZakaiBastien2017}. Det är enligt \textcite{Moller2018} \emph{inte} ett uttalat mål att WebAssembly skall ersätta JavaScript, utan tanken är att de skall \emph{komplettera} varandra. Ett exempel på hur JavaScript och WebAssembly skulle kunna komplettera varandra är att stora delar i populära JavaScript-ramverk skulle kunna bytas ut mot WebAssembly samtidigt som dess gränssnitt mot användaren av ramverket skulle kunna bibehållas helt oförändrade.

WebAssembly är en relativt ny teknik inom webbteknologi. De första artiklarna om WebAssembly publicerades 2017 \parencite{HaasRossbergSchuffTitzerHolmanGohmanWagnerZakaiBastien2017,ReiserBlaser2017}.

\section{Method}
\label{method}

To answer the research question a technology-oriented experiment will be conducted \parencite{WohlinRunesonHostOhlssonRegnellWesslen2012}. According to \textcite{WohlinRunesonHostOhlssonRegnellWesslen2012} an experiment is appropriate to precisely and systematically measure something in a controlled environment. Another approach would be to perform a case study \parencite{WohlinRunesonHostOhlssonRegnellWesslen2012}, but this method falls short on the fact that as of today few WebAssembly implementations exists. While an experiment has both execution control, measurement control, and is easy to replicate the investigation cost is higher compared to a survey or a case study and requires more resources \parencite{WohlinRunesonHostOhlssonRegnellWesslen2012}. In practice this means that access to a set of computing devices is a required prerequisite for this thesis. There is also a time investment to be done to write and execute all use cases. The dependent variables is the execution time in milliseconds. Other dependent variables could have been the size on disk of the compilation target, or the size of memory consumed while executing. The choice of execution time was made based on the notion that execution time is the most important variable for the end user. There are a number of independent variables: compilation target, web browser, operating system, and computing platform.

This thesis aims to implement a technical artifact in the form of a benchmarking suite that contain a set of use cases with different characteristics. The use cases are aimed to be representing common tasks \parencite{WohlinRunesonHostOhlssonRegnellWesslen2012} in modern web applications such as compression, encryption, sorting, floating point calculations, regular expressions, 2D and 3D rendering. Some use cases such as DOM manipulation can unfortunately not be tested as they are not yet supported by WebAssembly. The runtime of the use cases will be executed and compared on all major browsers on all major operating systems on all major computing platforms. Each use case will be an implementation of an algorithm chosen to test a specific aspect of WebAssembly and will be compiled from C/C++ to both WebAssembly and asm.js using Emscripten. The suite of compiled tests will be published as open source on Github. The benchmarking suite will be executed on the latest versions of the web browsers Chrome, Edge, Firefox and Safari. The result will be statistically evaluated using ANOVA in order to answer the research question.

\subsection{Ethics}

This experiment does not include any other humans than the author pressing the button to start the experiment. WebAssembly och asm.js are both open specifications. Emscripten is open source, available on Github. The forthcoming benchmarking suite will be open source and available on Github. The results along with this thesis is available on Github. WebAssembly and JavaScript is executed in sandboxed environment which means that even if there would be nefarious code in the benchmarking suite running the benchmarking suite would not be able to spread a computer virus or spy on the computer on which it was executed. The benchmark suite will be implemented based on best practices for implementing use cases such as the guidelines outlined by \textcite{CaiNerurkarWu1998} in an effort to provide as reliable results as possible.

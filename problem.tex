\section{Problem}

The problem is Benchmarking WebAssembly. The research question is to determine if there is a significant performance difference between WebAssembly and JavaScript in the web browsers Chrome, Edge, Firefox and Safari on Windows and macOS.

%According to X WebAssembly is "faster". 
%Are there any particular algorithms that WebAssembly is faster than?
%Is there a difference between major web browsers?

\subsection{Hypothesis}

The hypothesis is that there is a significant difference ($\alpha = 0,01$) in performance between WebAssembly and JavaScript.

\subsection{Previous work}

WebAssembly performance has been measured before. \textcite{HaasRossbergSchuffTitzerHolmanGohmanWagnerZakaiBastien2017} use the Polyhedral Benchmark suite\footnote{http://web.cs.ucla.edu/~pouchet/software/polybench/} (PolyBench/C) to compare the execution time between native code and WebAssembly. They use Emscripten to generate WebAssembly and Clang \parencite{LattnerAdve2014} to compile regular application and then compares the two \parencite{HaasRossbergSchuffTitzerHolmanGohmanWagnerZakaiBastien2017}. \textcite{ReiserBlaser2017} use their own JavaScript-implementation, that tests wether the input is a prime or not, to compare the execution time between JavaScript and WebAssembly. They use their own compiler (speedy.js) to compile the same JavaScript-code to WebAssembly and compare the execution time between WebAssembly and JavaScript.

\textcite{JangdaPowersGuhaBerger2019} use BROWSIX to run unmodified unix application inside the browser.

It seems that no one has previously investigated the difference between WebAssembly and JavaScript using one of the test suites that is traditionally used to test current JavaScript engines.
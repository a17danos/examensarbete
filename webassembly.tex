can also ship a program that acts as a virtual machine and thus run other languages in webassembly today

''WebAssembly is a new low-level language currently being implemented in all major web browsers. It is designed to become the universal compilation target for the web, obsolet- ing existing solutions in this area, such as asm.js and Native Client.'' \parencite{Watt2018}

WebAssembly\footnote{https://webassembly.org/} (Wasm) is by definition according to \textcite{HaasRossbergSchuffTitzerHolmanGohmanWagnerZakaiBastien2017} a portable assembly-language for web browsers and future adaptations. Technically WebAssembly is bytecode created by \emph{compiling} (any form of) code to WebAssembly \parencite{Watt2018} which then can be executed in any WebAssembly-engine. This can be compared with JavaScript-code that is \emph{interpreted} while running inside a JavaScript-engine.

WebAssembly can be seen as a successor to earlier technologies such as asm.js from Mozilla and Native Client (NaCL) from Google. NaCL executes native code in a separate part of Chrome and asm.js  \parencite{Zakai2018} is a subset of JavaScript optimized for performance \parencite{VanEsNicolayStievenartDHondtDeRoover2016} and can be interpreted in any web browser.

Earlier work on what today is WebAssembly has also resulted in EmScripten \parencite{Zakai2011}, a compiler based on LLVM \parencite{LattnerAdve2014} that originates as a transpiler from JavaScript to asm.js \parencite{Zakai2011} that has been further developed \parencite{HaasRossbergSchuffTitzerHolmanGohmanWagnerZakaiBastien2017} and is now able to compile both JavaScript and C/C++ to both asm.js and WebAssembly. EmScripten is the most common compiler to compile WebAssembly.

WebAssembly is the result of joint research and development between Apple, Google, Microsoft and Mozilla \parencite{HaasRossbergSchuffTitzerHolmanGohmanWagnerZakaiBastien2017}. WebAssembly has as the first technology since JavaScript full support in Chrome, Edge, Firefox and Safari. Those web browsers that does not support WebAssembly can according to \textcite{HaasRossbergSchuffTitzerHolmanGohmanWagnerZakaiBastien2017} use asm.js as ''polyfill''. WebAssembly is already being used where performance is of high importance, such as to generate cryptocurrency \parencite{RuthZimmermannWolsingHohlfeld2018}.

Initially WebAssembly supports code written in C/C++ \parencite{HaasRossbergSchuffTitzerHolmanGohmanWagnerZakaiBastien2017}. 
The focus on C/C++ is largely based on WebAssembly implementation limitations such as the lack of garbage collection. According to \textcite{HaasRossbergSchuffTitzerHolmanGohmanWagnerZakaiBastien2017} its a highly prioritized goal to allow WebAssembly to gain to access the web browsers built in garbage collector and that way support languages that use garbage collection.

WebAssembly is loaded as a module via a JavaScript API or another WebAssembly module \parencite{HaasRossbergSchuffTitzerHolmanGohmanWagnerZakaiBastien2017}. It's \emph{not} a goal to have WebAssembly replace JavaScript\footnote{https://webassembly.org/docs/faq/\#is-webassembly-trying-to-replace-javascript}, the idea is that they \emph{complement} each other. One example of how JavaScript and WebAssembly could complement each other is to replace large portions of JavaScript within popular JavaScript-frameworks with WebAssembly, but keep the JavaScript API towards the developer and thus only provide the benefit of performance.

WebAssembly is new technology. The first articles about WebAssembly was published in 2017 \parencite{HaasRossbergSchuffTitzerHolmanGohmanWagnerZakaiBastien2017,ReiserBlaser2017}.

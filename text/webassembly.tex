WebAssembly\footnote{\url{https://webassembly.org/}} (Wasm) is described by \textcite{Watt2018,JangdaPowersGuhaBerger2019} as a low-level language designed to become a universal compilation target for the web. According to the original authors \textcite{HaasRossbergSchuffTitzerHolmanGohmanWagnerZakaiBastien2017} it is a portable assembly-language for web browsers and future adaptations. Technically WebAssembly is bytecode created by \emph{compiling} (any form of) code to WebAssembly \parencite{Watt2018} which then can be executed in any WebAssembly-engine. This can be compared with JavaScript-code that is \emph{interpreted} while running inside a JavaScript-engine.

WebAssembly can be seen as a successor to earlier technologies such as asm.js from Mozilla (previously discussed) and Native Client (NaCL) from Google. NaCL executes native code in a separate part of Chrome and asm.js  \parencite{Zakai2018} is a subset of JavaScript optimized for performance \parencite{VanEsNicolayStievenartDHondtDeRoover2016} and can be interpreted in any web browser. Emscripten \parencite{Zakai2011}, a compiler based on LLVM \parencite{LattnerAdve2014} that originates as a source-to-source compiler from JavaScript to asm.js \parencite{Zakai2011} has been further developed \parencite{HaasRossbergSchuffTitzerHolmanGohmanWagnerZakaiBastien2017} and is now able to compile to both asm.js \emph{and} WebAssembly.

WebAssembly is the result of joint research and development between Apple, Google, Microsoft and Mozilla \parencite{HaasRossbergSchuffTitzerHolmanGohmanWagnerZakaiBastien2017}. WebAssembly has as the first technology since JavaScript full support in Chrome, Edge, Firefox and Safari and could as such be described as the forth core web technology (Figure \ref{webtechnologies-timeline}). Those web browsers that does not support WebAssembly can use asm.js as polyfill \parencite{HaasRossbergSchuffTitzerHolmanGohmanWagnerZakaiBastien2017}. WebAssembly is already being used where performance is of high importance, such as to generate cryptocurrency \parencite{RuthZimmermannWolsingHohlfeld2018}.

Initially WebAssembly supports code written in C/C++ \parencite{HaasRossbergSchuffTitzerHolmanGohmanWagnerZakaiBastien2017}. 
The focus on C/C++ is largely based on WebAssembly implementation limitations such as the lack of garbage collection. According to \textcite{HaasRossbergSchuffTitzerHolmanGohmanWagnerZakaiBastien2017} its a highly prioritized goal to allow WebAssembly to gain to access the web browsers built in garbage collector and that way support languages that use garbage collection.

WebAssembly is loaded as a module via a JavaScript API or another WebAssembly module \parencite{HaasRossbergSchuffTitzerHolmanGohmanWagnerZakaiBastien2017}. It is \emph{not} a goal to have WebAssembly replace JavaScript\footnote{\url{https://webassembly.org/docs/faq/\#is-webassembly-trying-to-replace-javascript}}, the idea is that they \emph{complement} each other. One example of how JavaScript and WebAssembly could complement each other is to port selected parts of a popular framework or web app for improved performance without changing the interface.

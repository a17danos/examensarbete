\section{Introduction}
\label{introduction}

The web was invented to enable information sharing between remote collaborators \parencite{BernersLeeCailliauLuotonenNielsenSecret1994}. JavaScript was intended to be used for simple tasks such as client-side form validation \parencite{Moller2018}, but lead the evolution of the web browser as a web page reader towards a platform for web applications.

WebAssembly is a portable assembly-language for the web \parencite{HaasRossbergSchuffTitzerHolmanGohmanWagnerZakaiBastien2017} and can be seen as the forth core web technology. It is the first web technology since 1996 to be universally adopted by all major web browser vendors \parencite{HaasRossbergSchuffTitzerHolmanGohmanWagnerZakaiBastien2017}.

Replacing certain type of JavaScript with WebAssembly raises the question of how to Benchmark WebAssembly. It is important to be aware of how benchmarking work and try to avoid mistakes such as dead code, known data sets and shortcuts \parencite{CaiNerurkarWu1998}.

This thesis makes the following contributions:

\begin{itemize}
    \item Implementation of a number of relevant use cases as both asm.js and WebAssembly,
    \item provided in the form of a open source benchmarking suite,
    \item executed on viable combinations of web browsers, operating systems and computing devices,
    \item analyzed and evaluated to provide guidelines when porting JavaScript to WebAssembly.
\end{itemize}

The rest of the thesis is organised as follows. Chapter \ref{background} summarizes core web technologies and discusses benchmarking. Chapter \ref{problem} describes the research problem and previous work. Chapter \ref{method} outlines the method and research ethics. 
%The implementation is described in chapter \ref{implementation}, and the results and analysis is presented in chapter \ref{evaluation}. Chapter \ref{conclusion} concludes the thesis.
